\documentclass{beamer}
\title{What Models Say}
\author{Hans Halvorson}
\date{October 27, 2023}
\providecommand{\tightlist}{%
  \setlength{\itemsep}{0pt}\setlength{\parskip}{0pt}}
\usepackage{tikz}
\usepackage{tikz-cd}
\tikzcdset{scale cd/.style={every label/.append style={scale=#1},
    cells={nodes={scale=#1}}}}

\usetikzlibrary{calc}
\usepackage{pgfplots}
\usetheme{Copenhagen}
\useoutertheme{infolines}
\setbeamertemplate{navigation symbols}{}
% \usepackage{quiver}
\begin{document}

\begin{frame}[plain]

  \titlepage

\end{frame}

\section{Introduction}

\newcommand{\e}{\emptyset}

\begin{frame}{An example}

\[ M \: = \: \Bigl\langle \{ a,\langle a,a\rangle  \} ,\{ \langle a,a\rangle \}
  \Bigr\rangle \]

\bigskip 

\begin{itemize}
\item What does $M$ say?
    \begin{enumerate}
    \item There are two things, and there is a relation that holds
      between the first thing and itself.
    \item There are two things, and there is a property that holds of
      the second thing.
    \end{enumerate}
  \item The mathematical structure is ambiguous, and so it doesn't say
    anything.
    \begin{itemize}
    \item Perhaps we should consider (1) and (2) to be ``predicate
      precisifications'' of $M$? (Wallace 2022)
    \end{itemize}
  \end{itemize}

\end{frame}

\begin{frame}{Using models}

  \begin{itemize}
  \item Fact: Contemporary physics uses (mathematical) models to
    represent physical reality.
    
  \item Truism: Physicists often can explain in words how they intend
    for their mathematical models to represent reality.

    \begin{itemize}
    \item When they cannot, then they are instrumentalists.
    \item Does high energy physics show that discursive understanding
      is no longer the goal of model building?
    \item Disclaimer: I'm not qualified to speak about normal practice
      in physics.
    \end{itemize}
  \end{itemize}  

  \end{frame}

\begin{frame}{Using models}

  \begin{itemize}
  \item From language to model: physicists use language to specify the
    relevant class of models and/or to identify the representational
    relevant parts of models.
  \item From model to language: physicists use models to make
    statements about the physical world.
  \end{itemize}

\end{frame}

\begin{frame}{Using models}

  \begin{itemize}
  \item Semantic view: To accept a theory is to believe that one if
    its models represents the intended object or system.
  \item ``$M$ represents reality'' is not specific enough.
    \begin{itemize}
    \item Example: ``$42$ represents reality.''
    \item Example: ``$\mathbb{R}$ represents reality.''
    \end{itemize}
  \end{itemize}

\end{frame}

\begin{frame}{Artifact versus Content}

  ``It's fine to construct models with artifacts. But there must
  always be some way of describing the phenomenon in question that (in
  some sense) lacks artifacts. There must be some way of saying what
  is really going on. For example, although we can model mass with
  real numbers, there must be some underlying artifact-free
  description, such as the $\succeq$ and $C$ description, from which
  one can recover a specification of which numerical models are
  acceptable, and a specification of which features of the models are
  artifacts.''  (Sider 2020, p 192)

\end{frame}

  \section{Three examples}

\begin{frame}{Features of the wavefunction}

  \begin{quote}
    ``On the Ghirardi-Rimini-Weber (GRW) theory (or, for that matter,
    on any theory of collapse), the world will consist of exactly one
    physical object---the universal wave function. What happens, all
    that happens, is that the function changes its shape in accord
    with the theory's dynamical laws.''
  \end{quote}
\end{frame}

\begin{frame}{Features of the wavefunction}

  \begin{itemize}
  \item Which \textbf{features} of a wavefunction belong to its
    representational content, and which features are representational
    artifacts?
  \item Wavefunction realists should grant that some features of the
    wavefunction are artifacts.

  \begin{itemize}
  \item ``$\psi$ is written in computer modern font'' is an artifact.
  \item ``$\{ \emptyset \}$ is in the domain of $\psi$'' is an
    artifact.
  \end{itemize}
\end{itemize}

\end{frame}

\begin{frame}{Features of the wavefunction}

\begin{itemize}
\item Real features of $\psi$

  \begin{itemize}
  \item Expectation values
    \begin{itemize}
    \item But: even expectation values are not invariant under all
      isomorphisms, e.g.~the isomorphism that takes ``position is
      represented by the operator $Q$'' to ``position is represented
      by the operator $Q+aI$''.
    \end{itemize}
  \end{itemize}

\item Pseudo-features of $\psi$

  \begin{itemize}
  \item $\psi$ is a subset of $\mathbb{R}^{3n}\times \mathbb{C}$, and
    there is a $c\in\mathbb{C}$ such that
    $\langle \emptyset ,c\rangle\in \psi$.
  \item $\psi$ is represented by ``$\psi$''
  \end{itemize}
\end{itemize}

\end{frame}

\begin{frame}{Features of the wavefunction}

\begin{itemize}
\item Possibly real features of $\psi$

  \begin{itemize}
  \item The value of $\psi$ at a particular point $a\in X$

    \begin{itemize}
    \item What about Lebesque measure zero sets?
    \item What about $U(1)$ gauge freedom?
    \end{itemize}
  \item The value of $d\psi /dx$ at a particular point $a\in X$
  \end{itemize}
\end{itemize}
\end{frame}

\begin{frame}{Features of a group}

  \begin{itemize}
  \item Real features of $G$

    \begin{itemize}
  \item Cardinality
  \item Abelian or non-abelian
  \item Cyclic of order $n$
  \end{itemize}
\item Pseudo-features of $G$

  \begin{itemize}
  \item $G$ contains some particular element $a$.
  \item $G=G$ and the continuum hypothesis is true.
  \end{itemize}
\end{itemize}
\end{frame}

\begin{frame}{Features of a group}

\begin{itemize}
\item Real features of $G$: liberalized

  \begin{itemize}
  \item $G$ has $n$ normal subgroups.
  \item $G$ has $2$ irreducible representations.
  \end{itemize}
\end{itemize}
\end{frame}

\begin{frame}{Features of a spacetime}

\begin{itemize}
\item Pseudo-features of $(M,g_{ab})$
  \begin{itemize}
  \item The value of the scalar curvature at a particular point
    $a\in M$
  \end{itemize}

\item Real features of $(M,g_{ab})$
  \begin{itemize}
  \item The scalar curvature has no upper bound.
  \end{itemize}
\end{itemize}

\begin{itemize}
\item Possibly real features of $(M,g_{ab})$
  \begin{itemize}
  \item Inextendible?
  \end{itemize}

\end{itemize}
\end{frame}

\begin{frame}{Note}

  \begin{itemize}
\item Do physicists sometimes take models with different features to
  be representationally equivalent?
\item Yes, if we take ``features'' in a liberal sense.

  \begin{itemize}
  \item Example: distinct groups that are isomorphic
  \end{itemize}
\item Is there a more refined sense of ``features'' such that models
  are representationally equivalent only if they have the same
  features?

  \begin{itemize}
  \item Example: if $T_1$ and $T_2$ are Morita equivalent, then
    for each model $M$ of $T_1$ there is a \textbf{Morita twin}
    model $N$ of $T_2$.
  \item Example: Hamiltonian and Lagrangian mechanics
  \item Example: Spacetimes and Einstein algebras
  \end{itemize}
\end{itemize}
\end{frame}

\begin{frame}{Proposals}

\begin{itemize}
\item If a property $\Phi$ of $M$ is real, then $\Phi$ is mathematical
  (in some sense to be made precise).
\item If a property $\Phi$ of $M$ is real, then $\Phi$ is invariant
  under isomorphisms.
  \begin{itemize}
  \item Problem: Which isomorphisms? Two mathematical objects can be
    isomorphic relative to one category and non-isomorphic relative to
    another category.
    \begin{itemize}
    \item Example: Minkowski versus FRW spacetime.

    \end{itemize}
  \end{itemize}
\end{itemize}
\end{frame}

\section{Features of models}

\begin{frame}{Overview}

\begin{itemize}
\item Even for a first-order theory $T$, it's not clear what counts as
  a genuine feature of a model $M$ of $T$.

  \begin{itemize}
  \item Signature-free accounts
    \begin{itemize}
    \item Set-theoretic
    \item Category-theoretic
    \end{itemize}
  \item Signature-relative accounts
    \begin{itemize}
    \item Elementary properties
    \item Second-order properties
    \item $\vdots$   
    \item Väänänen's sort logic (designed to account for relational
      properties with other mathematical objects)
    \end{itemize}
  \end{itemize}
\end{itemize}  


\end{frame}


\begin{frame}{Set-theoretic properties of models}

  \begin{itemize}
  \item Suppes' proposal: If $\Phi$ is a predicate in the language of
    ZF set theory such that $ZF\vDash\Phi (M)$, then $\Phi$ represents
    a property of $M$.
  \item Reply: Physicists don't care about most set-theoretic
    properties.
    \begin{itemize}
    \item Physicists aren't concerned about whether $M$ contains
      $\emptyset$, or $\{ \emptyset \}$, or $\{ \{ \emptyset \} \}$,
      etc.
    \item ``$M=M$ and the continuum hypothesis is true'' is an
      isomorphism-invariant property of $M$.
      \begin{itemize} \item The problem is not that $\Phi$ is
        extrinsic to $M$, because physicists do care about some
        extrinsic properties of their models!
      \end{itemize}
    \end{itemize}
  \end{itemize}

\end{frame}

\begin{frame}{Category-theoretic properties of models}

  \begin{itemize}
  \item Idea 1: A model's relations to other models is an important
    aspect of its use in physics.

    \begin{itemize}
    \item Example: Inextendible spacetimes
      \begin{itemize}
      \item But is it inextendible because of something about its
        internal structure?
      \end{itemize}
    \end{itemize}

  \item Idea 2: A model's symmetries are an important aspect of its
    use in physics.

    \begin{itemize}
    \item Example: In EM, how we understand the content of a model
      depends on what we take to be isomorphisms between models.
    \end{itemize}
  \end{itemize}

\end{frame}

\begin{frame}{Category-theoretic properties of models}

\begin{itemize}
\item Category theory is a first-order theory with two sort symbols
  $O$ and $A$, two function symbols $d_0,d_1:A\to O$, a function
  symbol $1:O\to A$, and a partial function symbol
  $\circ :A\times A\to A$.
\item A model $\mathbf{C}$ of category theory consists of two sets
  $\mathbf{C}_0$ and $\mathbf{C}_1$.
\item If we take a property of $a\in \mathbf{C}_0$ to be a sentence
  $\phi (x)$ such that $\mathbf{C}\models \phi (a)$, then the typical
  property of objects are relational features such as:

  \begin{itemize}
  \item Being embeddable in certain other kinds of objects.
  \item Having a certain number of automorphisms.
  \end{itemize}
\end{itemize}
\end{frame}

\begin{frame}{Signature-dependent accounts}

  \begin{itemize}
  \item Let $\Sigma$ be a signature, and let $M$ be a
    $\Sigma$-structure.
  \item FO sentences yield properties: $M$ has property $\phi$ iff
    $M\vDash \phi$.
    \begin{itemize}
    \item Example: ``$M$ has two elements'' is true just in case
      $M\vDash \exists _{=2}(x=x)$.
    \end{itemize}
    \item Fact: These elementary properties are isomorphism invariant.
    \end{itemize}
  \end{frame}

\begin{frame}{Signature-dependent accounts}

  \begin{itemize}
  \item Typically there are isomorphism-invariant properties that are
    not expressible by first-order sentences.
  \item Some are expressible by second-order sentences.
    \begin{itemize}
    \item Example: ``$M$ is uncountably infinite.''
    \item Example: ``$M$ is a compact topological space.''
    \end{itemize}
  \item Some are not expressible by second-order sentences.
    \begin{itemize}
    \item Example: ``$M$ can be embedded in $\mathbb{R}^4$.''
    \end{itemize}
  \end{itemize}
  
\end{frame}


\begin{frame}{Tentative conclusions}

  \begin{itemize}
  \item There are too many set-theoretic properties of models.
  \item There are too few category-theoretic properties of models.
  \item Any good account of the properties of models will make these
    properties signature-dependent.
  \end{itemize}


\end{frame}


\begin{frame}{Application to examples}

  \begin{itemize}
  \item Relativity: Lorentzian geometry may not be first-order, but
    there is a sense in which it is a linguistically formulated
    theory.
    \begin{itemize}
    \item There is progress to be made on isolating the relevant
      features of models.
    \end{itemize}
  \item Quantum: Any two wavefunctions are related by a unitary
    symmetry. Thus, by standard accounts, no wavefunction can have
    features that another lacks.
    \begin{itemize}
    \item What wavefunction realists need is an account of
      basis-relative features --- i.e.\ features of elements of
      $L_2(X)$ or $\ell _2(X)$.
    \end{itemize}
  \end{itemize}

\end{frame}

\section{Conclusions}

\begin{frame}{Conclusions}

\begin{itemize}
\item I think that ``artifact free representation'' is a will o' the
  wisp.
  \begin{itemize}
  \item The goal of my suggested ``program'' is not --- like Hartry
    Field's --- to eliminate representational artifacts from models.
  \item The goal is rather to understand the subtle practice of
    distinguishing relevant (intended) from irrelevant (unintended)
    features of models.
  \end{itemize}
\item The dilemma for language-free accounts of the features of
  models:
  \begin{itemize}
  \item If models are sets, then they have more properties than
    physics cares about.
  \item If models are objects in a category, then they have fewer
    properties than physics cares about.
  \end{itemize}
\end{itemize}

\end{frame}


\begin{frame}{Conclusions}

  \begin{itemize}
  \item Believing that $M$ has artifacts is tantamount to rejecting
    the unqualified statement that $M$ represents the world.
  \item We use $M$ to generate sentences that describe the world.
  \item Perhaps mathematics first, but not only mathematics.
  \end{itemize}

\end{frame}  



%% TO DO: McLarty on the natural numbers

%% TO DO: the ambiguous model 


\end{document}

\begin{frame}{Practical questions}

\begin{itemize}
\item What do we take to be satisfactory theoretical achievement in
  physics

  \begin{itemize}
  \item Are we looking for \textbf{the one true model}?
  \item Is the situation suboptimal if there are two distinct
    representations of the same situation?
  \item Is the situation suboptimal if we have a higher-level theory
    that cannot be syntactically reduced to a more fundamental theory?
  \end{itemize}
\end{itemize}
\end{frame}

\begin{frame}{Practical questions}

\begin{itemize}
\item Does a solution to the measurement problem require ``new
  physics''?

  \begin{itemize}
  \item
    The answer depends in part on whether macro-reality is
    \textbf{reducible} to the wavefunction.
  \item
    Idea: Bohmians have a language-first notion of reduction.
  \end{itemize}
\end{itemize}
\end{frame}

\begin{frame}{The language-first police}

  \begin{quote}
``A physical theory should clearly and forthrightly address two
fundamental questions: what there is, and what it does. The answer to
the first question is provided by the ontology of the theory, and the
answer to the second by its dynamics. The ontology should have a sharp
mathematical description, and the dynamics should be implemented by
precise equations describing how the ontology will, or might, evolve.''
\end{quote}
\end{frame}

\begin{frame}{Between language and mathematics}
\protect\hypertarget{between-language-and-mathematics}{}
\begin{itemize}
\item
  The ``language first'' and ``math first'' views get closer when we
  realize that:

  \begin{enumerate}
  \item
    Linguistic notions can be made more precise and more flexible.

    E.g. translation schemes do not need to match simple object
    descriptors to simple object descriptors.
  \item
    Precise mathematical notions are usually expressible in language.

    E.g. sentences about mathematical objects are preserved under
    isomorphism.
  \end{enumerate}
\end{itemize}
\end{frame}

\begin{frame}{Intro}
\protect\hypertarget{intro}{}
\begin{itemize}
\item
  The relation between \textbf{model} and \textbf{world} was hoped to be
  more direct and transparent than the relation between language and
  world.
\item
  This hope proved to be naive.
\end{itemize}

\begin{enumerate}
\tightlist
\item
  The model $M$ is intended to be isomorphic with the represented
  object.
\end{enumerate}
\end{frame}

\begin{frame}{Model isomorphism criterion}
\protect\hypertarget{model-isomorphism-criterion}{}
\begin{itemize}
\item
  Theories $T_1$ and $T_2$ are equivalent only if the models of the
  two correspond one-to-one, where correspondence maintains isomorphism.
\item
  The model isomorphism criterion is appropriate for those who believe
  that the goal is to find a model that is isomorphic with the
  represented object.
\item
  Against: Even conservative accounts have that theories in different
  languages can be equivalent. But models in different languages cannot
  be isomorphic.
\item
  For: Moderate views of theoretical equivalence entail that if $T_1$
  is equivalent to $T_2$, then for each model $M_1$ of $T_1$,
  there is a twin model of $M_2$, and these models can be constructed
  from each other.
\end{itemize}
\end{frame}

\begin{frame}{Conceptual housekeeping}
\protect\hypertarget{conceptual-housekeeping}{}
\begin{itemize}
\item
  Proposal: As far as physics practice goes, a mathematical object $M$
  has no features in itself.

  \begin{itemize}
  \item
    A mathematical object has properties \emph{qua} member of a category
    \(\mathbf{C}$.

    \begin{itemize}
    \tightlist
    \item
      Properties of ``\(M$ in \(\mathbf{C}$'' are invariant under
      isomorphisms in \(\mathbf{C}$.
    \end{itemize}
  \item
    When physicists use \(M$, they keep mental note of the background
    category \(\mathbf{C}$.
  \end{itemize}
\end{itemize}
\end{frame}

\begin{frame}{Conceptual housekeeping}
\begin{itemize}
\item
  Example: Despite the familiar notation, \(\mathbb{R}$ is not a well
  defined mathematical object.

  \begin{itemize}
  \tightlist
\item It won't help to say: $\mathbb{R}$ is a generic name for any set
  in a certain isomorphism class.
  \end{itemize}
\item
  Application: The notion of \textbf{reduction} is category relative.


  \subsection{Semantics versus Syntax}
\item Idea: Used in the right way, mathematics can increase the
  objectivity of our descriptions. Used in the wrong way, mathematics
  obscures our descriptions.

  \begin{itemize}
  \item
    ``There were 42 people'' is more objective than ``there were a lot
    of people.''
  \item
    ``The world is modelled by 42'' is obscure.
  \end{itemize}
\item
  Much of mathematics is ``new language'' in the sense of David Lewis.

  \begin{itemize}
  \item
    New mathematical objects don't \emph{say} anything until we connect
    them to old language.
  \item
    Mathematical models don't \emph{represent} anything.
  \end{itemize}
\end{itemize}
\end{frame}

\end{document}
%%% Local Variables:
%%% mode: latex
%%% TeX-master: t
%%% End:
